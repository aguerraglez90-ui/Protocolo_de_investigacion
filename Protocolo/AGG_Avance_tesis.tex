\documentclass[12pt,letterpaper]{article}

% =========================
% PAQUETES BÁSICOS
% =========================
\usepackage[spanish]{babel}
\usepackage[utf8]{inputenc}
\usepackage[T1]{fontenc}
\usepackage{lmodern}
\usepackage{geometry}
\geometry{left=3cm,right=2.5cm,top=3cm,bottom=3cm}
\usepackage{setspace}
\onehalfspacing

% =========================
% MATEMÁTICAS
% =========================
\usepackage{amsmath,amssymb,amsfonts}
\usepackage{bm}
\usepackage{physics}

% =========================
% FIGURAS Y TABLAS
% =========================
\usepackage{graphicx}
\usepackage{float}
\usepackage{booktabs}

% =========================
% CÓDIGO
% =========================
\usepackage{listings}
\usepackage{xcolor}

\definecolor{codegray}{rgb}{0.5,0.5,0.5}
\definecolor{codeblue}{rgb}{0.0,0.2,0.6}

\lstset{
	language=Python,
	basicstyle=\ttfamily\small,
	keywordstyle=\color{codeblue},
	commentstyle=\color{codegray},
	numbers=left,
	numberstyle=\tiny,
	stepnumber=1,
	numbersep=5pt,
	frame=single,
	breaklines=true,
	tabsize=2
}

% =========================
% BIBLIOGRAFÍA
% =========================
\usepackage[numbers]{natbib}

% =========================
% DATOS DEL DOCUMENTO
% =========================
\title{\textbf{Protocolo  de Investigacion}}
\author{
	Nombre del Autor \\
	\small Posgrado en Ciencias en Ingeniería Química \\
	\small Universidad Autónoma de San Luis Potosí
}
\date{\small \today}

\begin{document}
	\maketitle
	
	% =========================
	% RESUMEN
	% =========================
	\begin{abstract}
		En este trabajo se presenta la formulación matemática, el análisis dinámico y la implementación computacional de sistemas descritos mediante ecuaciones diferenciales ordinarias y ecuaciones de reacción--difusión. Se estudian los supuestos físicos involucrados, la estabilidad de soluciones y la resolución numérica mediante métodos estándar. La implementación en Python permite reproducir los resultados y analizar escenarios dinámicos relevantes para sistemas químicos y biológicos.
	\end{abstract}
	
	\noindent \textbf{Palabras clave:} ecuaciones diferenciales, reacción--difusión, modelación matemática, simulación numérica.
	
	% =========================
	\section{Introducción}
	% =========================
	Segun la Organizacion Mundial de la Salud (OMS), se denomina cancer a un amplio grupo de enfermedades que pueden afectar a cualquier parte del organismo; también se habla de «tumores malignos» o «neoplasias malignas». Una característica definitoria del cáncer es la multiplicación rápida de células anormales que se extienden más allá de sus límites habituales y pueden invadir partes adyacentes del cuerpo o propagarse a otros órganos, en un proceso que se denomina «metástasis». En 2020 fue la primera causa de muerte a nivel mundial, llegando a un numero total de casi 10 millones de defunciones por esta enfermedad. Los cánceres más comunes en 2020, por lo que se refiere a incidencia, fueron los siguientes:
	
	\begin{itemize}
		\item mama (2,26 millones de casos);
		\item pulmón (2,21 millones de casos);
		\item colorrectal (1,93 millones de casos);
		\item próstata (1,41 millones de casos);
		\item piel (distinto del melanoma) (1,20 millones de casos); y
		\item gástrico (1,09 millones de casos).
	\end{itemize}
	
	Los tipos de cáncer que causaron un mayor número de fallecimientos en 2020 fueron los siguientes:
	
	\begin{itemize}
		\item pulmón (1,8 millones de defunciones);
		\item colorrectal (916\,000 defunciones);
		\item hepático (830\,000 defunciones);
		\item gástrico (769\,000 defunciones); y
		\item mama (685\,000 defunciones).
	\end{itemize}
	

	Cada año, cerca de 400 000 niños contraen cáncer. Aunque los tipos de cáncer más frecuentes varían en función del país, el de cuello uterino es el más habitual en 23 países.
	
	El cáncer al ser una enfermedad multifactorial que surge como resultado de alteraciones coordinadas en múltiples niveles de organización biológica. El enfoque de redes complejas permite interpretar el cáncer como un fenómeno sistémico, más que como la consecuencia de mutaciones aisladas.
	
	Una \textit{red compleja} es una representación matemática de un sistema compuesto por un conjunto de elementos discretos, denominados \textbf{nodos}, y las interacciones entre ellos, llamadas \textbf{aristas}. Formalmente, una red puede describirse como un grafo 
	
	\begin{equation} 
		G = (V, E), 
	\end{equation} 
	
	donde $V$ es el conjunto de nodos y $E$ el conjunto de enlaces. A diferencia de grafos regulares o aleatorios simples, las redes complejas presentan estructuras topológicas no triviales que emergen de la interacción colectiva de sus componentes.
	
	Las redes complejas reales suelen caracterizarse por propiedades distintivas, entre ellas se encuentran: distribuciones de grado altamente heterogéneas, frecuentemente de tipo \textit{scale--free} 
	
	\begin{equation} 
		P(k) \sim k^{-\gamma}, 
	\end{equation} 

	la presencia de nodos altamente conectados (\textit{hubs}), un elevado coeficiente de agrupamiento, y una organización modular y jerárquica. Estas propiedades reflejan la coexistencia de orden y desorden, así como la aparición de fenómenos colectivos no lineales.
	
	La ventaja principal del estudio de redes coplejas es que el comportamiento global no puede entenderse únicamente a partir de las propiedades individuales de sus componentes. Una de sus limitaciones mas importante es que las redes son representaciones estaticas de los sistemas. Por lo que para llegar a realizar un estudio dinamico se necesita llevar estas interacciones a un sistema de ecuaciones diferenciales ordinario.
		
	\subsection{Conexión con modelos dinámicos}
	
	El formalismo de redes complejas permite extender el análisis estructural hacia modelos dinámicos, por ejemplo mediante sistemas de ecuaciones diferenciales ordinarias acopladas por la matriz de adyacencia de la red: 
	\begin{equation}
		\dot{x}_i = f(x_i) + \sum_{j} A_{ij}\, g(x_i, x_j)
	\end{equation}
	 

	donde $x_i$ representa la actividad del nodo $i$ y $A_{ij}$ codifica la estructura de interacciones. Este enfoque posibilita el estudio de estabilidad, bifurcaciones y transiciones dinámicas relevantes para la progresión tumoral.
	
	El objetivo de este trabajo es formular rigurosamente ambos tipos de modelos, analizar sus propiedades matemáticas fundamentales e implementar su resolución numérica utilizando Python.
	
	% =========================
	\section{Marco Teórico}
	% =========================
	
	\subsection{Sistemas de ecuaciones diferenciales ordinarias}
	Un sistema dinámico continuo puede expresarse en forma general como:
	\begin{equation}
		\frac{d\bm{x}}{dt} = \bm{f}(\bm{x},t), \quad \bm{x}(0)=\bm{x}_0
	\end{equation}
	donde $\bm{x}(t)\in\mathbb{R}^n$ representa el vector de estados y $\bm{f}$ describe la cinética del sistema.
	
	En el contexto de reacciones químicas homogéneas, las ecuaciones adoptan la forma:
	\begin{equation}
		\frac{dC_i}{dt} = R_i(C_1,C_2,\dots,C_N), \quad i=1,\dots,N
	\end{equation}
	donde $C_i$ es la concentración de la especie $i$ y $R_i$ la velocidad de reacción correspondiente.
	
	\subsubsection{Análisis de estabilidad}
	Los puntos estacionarios $\bm{x}^*$ satisfacen:
	\begin{equation}
		\bm{f}(\bm{x}^*) = \bm{0}
	\end{equation}
	La estabilidad local se determina mediante el Jacobiano:
	\begin{equation}
		J_{ij} = \left.\frac{\partial f_i}{\partial x_j}\right|_{\bm{x}=\bm{x}^*}
	\end{equation}
	La naturaleza de los autovalores de $J$ define el comportamiento dinámico local del sistema.
	
	% =========================
	\subsection{Sistemas de reacción--difusión}
	% =========================
	Los sistemas de reacción--difusión incorporan transporte espacial mediante difusión:
	\begin{equation}
		\frac{\partial C_i}{\partial t} =
		D_i \nabla^2 C_i + R_i(\bm{C}), \quad i=1,\dots,N
	\end{equation}
	donde $D_i$ es el coeficiente de difusión de la especie $i$.
	
	\subsubsection{Condiciones iniciales y de frontera}
	El problema se completa con:
	\begin{align}
		C_i(\bm{x},0) &= C_{i,0}(\bm{x}) \\
		\nabla C_i \cdot \bm{n} &= 0 \quad \text{(condición de no flujo)}
	\end{align}
	
	\subsubsection{Inestabilidad de Turing}
	La aparición de patrones espaciales ocurre cuando un estado homogéneo estable frente a perturbaciones homogéneas se vuelve inestable frente a perturbaciones espaciales, lo cual se analiza mediante una descomposición modal del tipo:
	\begin{equation}
		C_i = C_i^* + \epsilon e^{\lambda t} \cos(kx)
	\end{equation}
	
	% =========================
	\section{Metodología Numérica}
	% =========================
	Para la resolución numérica de EDO se emplean métodos de integración temporal adaptativos basados en Runge--Kutta. En el caso de ecuaciones de reacción--difusión, se utiliza discretización espacial mediante diferencias finitas y esquemas explícitos o implícitos para el avance temporal.
	
	% =========================
	\section{Implementación en Python}
	% =========================
	
	\subsection{Sistema de EDO}
	El siguiente script resuelve un sistema cinético simple usando \texttt{SciPy}:
	
	\begin{lstlisting}
		import numpy as np
		from scipy.integrate import solve_ivp
		import matplotlib.pyplot as plt
		
		def model(t, y):
		k1, k2 = 1.0, 0.5
		A, B = y
		dA = -k1 * A
		dB = k1 * A - k2 * B
		return [dA, dB]
		
		t_span = (0, 10)
		y0 = [1.0, 0.0]
		
		sol = solve_ivp(model, t_span, y0, dense_output=True)
		
		t = np.linspace(0, 10, 200)
		y = sol.sol(t)
		
		plt.plot(t, y[0], label='A')
		plt.plot(t, y[1], label='B')
		plt.legend()
		plt.xlabel('Tiempo')
		plt.ylabel('Concentración')
		plt.show()
	\end{lstlisting}
	
	\subsection{Sistema de reacción--difusión unidimensional}
	\begin{lstlisting}
		Nx = 100
		L = 10.0
		dx = L / Nx
		dt = 0.01
		D = 0.1
		
		C = np.ones(Nx) + 0.01*np.random.rand(Nx)
		
		for n in range(1000):
		C[1:-1] += dt * (
		D * (C[2:] - 2*C[1:-1] + C[:-2]) / dx**2
		- C[1:-1]
		)
	\end{lstlisting}
	
	% =========================
	\section{Resultados y Discusión}
	% =========================
	Los resultados numéricos muestran una evolución temporal consistente con la cinética teórica esperada. En los sistemas de reacción--difusión se observa la amplificación de perturbaciones espaciales bajo condiciones específicas de parámetros, lo cual concuerda con el análisis lineal de estabilidad.
	
	Desde el punto de vista computacional, los métodos empleados presentan buena estabilidad para pasos de tiempo moderados, aunque los esquemas explícitos imponen restricciones de tipo CFL.
	
	% =========================
	\section{Ventajas y Limitaciones}
	% =========================
	Entre las principales ventajas del enfoque propuesto se encuentra su claridad conceptual y facilidad de implementación. Sin embargo, para sistemas rígidos o dominios espaciales complejos, se requieren métodos implícitos o esquemas de elementos finitos con mayor costo computacional.
	
	% =========================
	\section{Conclusiones}
	% =========================
	Se presentó una formulación matemática rigurosa de sistemas dinámicos descritos por ecuaciones diferenciales ordinarias y de reacción--difusión, junto con su implementación numérica en Python. El enfoque permite estudiar estabilidad, dinámica transitoria y formación de patrones, siendo aplicable a una amplia variedad de sistemas físico-químicos.
	
	% =========================
	\bibliographystyle{plainnat}
	\begin{thebibliography}{9}
		\bibitem{murray}
		J. D. Murray,
		\textit{Mathematical Biology},
		Springer, 2002.
		
		\bibitem{logan}
		J. D. Logan,
		\textit{Applied Mathematics},
		Wiley, 2014.
	\end{thebibliography}
	
\end{document}
